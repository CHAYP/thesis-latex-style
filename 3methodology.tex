\justify

\chaptertitle{MATERIALS AND METHODS}
\section{Materials}
\begin{enumerate}
  \item Acer Swift SF314
  \item Logitech G403 hero
  \item SteelSeries QcK mini
  \item AKG N20
\end{enumerate}

\section{Methods}
\subsection{Intra-group boundaries}
We first prove a structural property of intra-group classes.  The
following lemma shows that it is possible to separate one class from
the rest in the same group using only lower and upper thresholds.
This is independent of the number of classes in that group.

\begin{lemma}
  For any group $i\in [L]$, for any class $y\in G_i$, there exists
  reals $b_i\leq t_i$ such that for all $t\in[T]$ such that
  (1) when $y_t=y$, 
  \[
  b_i + \gamma \leq \langle u'_i,x_t\rangle \leq t_i - \gamma;
  \]
  and (2) when $g(y_t)=g(y)$ but $y_t\neq y$, either
  \[
  \langle x_t,u'_i\rangle \leq b_i - \gamma,
  \]
  or
  \[
  \langle x_t,u'_i\rangle \geq t_i + \gamma.
  \]
  \label{lemma:boundaries}
\end{lemma}
\begin{proof}{lemma \ref{lemma:boundaries}}
  Let $S_y = \{(x_j,y_j) : y_j=y, 1\leq j\leq T\}$ be the set of
  examples with label $y$.  Let $b_i=\min_{(x,y)\in S_y}\langle
  x,u'_i\rangle-\gamma$ and $t_i=\max_{(x,y)\in S_y}\langle
  x,u'_i\rangle+\gamma$.  The lemma follows from the definition of
  group weakly linear separability.
\end{proof}

\subsection{Group weakly linear separability}
We now define group weakly linear separability.
Let ${\mathcal G}=\{G_1,G_2,\ldots,G_L\}$ be a partition of $[K]$, i.e., $G_i\subseteq [K]$ for all $i$,
$G_i\cap G_j=\emptyset$ for $i\neq j$, and $\bigcup G_i = [K]$.  
Let $g:[K] \rightarrow [L]$ be a mapping function such that $g(i)\mapsto j$ iff $i\in G_j$.
We say that the labeled examples
$(x_1,y_1),(x_2,y_2),\ldots,(x_T,y_T)\in V\times[K]$
are {\em group weakly linear separable with margin $\gamma$ under ${\mathcal G}$} 
if 
\begin{enumerate}
\item there exist vectors $u_1,u_2,\ldots,u_L\in V$ such that
$\sum_{i=1}^L \Vert u_i \Vert^2\leq 1$, and, for all $t\in[T]$, 
\[
\langle x_t, u_{g(y_t)}\rangle \geq \langle x_t, u_p\rangle + \gamma,
\]
for all $p\in [L] \setminus\{g(y_t)\}$, 
%
\item there exist vectors $u'_1,u'_2,\ldots,u'_L\in V$ such that
$\sum_{i=1}^L \Vert u'_i \Vert^2\leq 1$, and, for all $t\in[T], t'\in[T]$ such that $y_t\neq y_{t'}$ and $g(y_t)=g(y_{t'})$,
either
\[
\langle x_t, u'_{g(y_t)}\rangle \geq \langle x_{t'}, u'_{g(y_t)}\rangle + 2\gamma,
\]
or
\[
\langle x_t, u'_{g(y_t)}\rangle \leq \langle x_{t'}, u'_{g(y_t)}\rangle - 2\gamma.
\]
\end{enumerate}

Note that vectors $u_i$'s define inter-group hyperplanes, while each
$u'_i$ defines intra-group boundaries.  Also note that, to simplify
our proofs, the ``margin'' between intra-group classes is $2\gamma$;
this would create the $+\gamma$ and $-\gamma$ gaps that already exist
between groups.

To illustrate the idea, Fig.~\ref{fig:sep-examples} showing different linear separable conditions.
Thick lines represent class boundaries. 
(a) Strongly linear separable examples with 3 classes (linearly separable in $\R^3$).  
(b) Weakly linear separable examples with 3 classes.  
(c) Group weakly linear separable examples with 3 groups; group 1 (white) contains 3 classes, group 2 (black) contains 4 classes, and group 3 (gray) contains 1 class.  

\begin{figure}[h]
\centering
\includegraphics[width=0.45\textwidth]{sep-examples.png}
\caption{Three set of examples in $\R^2$.}
\label{fig:sep-examples}
\end{figure}

\subsection{Margin transformation}

This section is devoted to the proofs of
Theorem~\ref{thm:margin-trans}.  A key property of the space $\ell_2$
is that it contains all multivariate polynomials and the rational
kernel $k$ allows us to work in that space.  The following lemma is
from~\cite{BeygelzimerPSTWZ2019-separable}.

\begin{lemma}[from Lemma 9 in~\cite{BeygelzimerPSTWZ2019-separable}]
\label{lem:norm-bound}
(Norm bound) Let $p:\R^d\to \R$ be a multivariate polynomial. There exists $c\in \ell_2$ such that $p(x)=\fdot{c}{\phi(x)}_{\ell_2}$ and $\|c\|_{\ell_2}\leq 2^{deg(p)/2}\|p\|.$
\end{lemma}

To proof Theorem~\ref{thm:margin-trans}, we need to establish the
existence of multivariate polynomials that separate one class from the
other.  Consider class $i\in [K]$ in group $g(i)$.  Its positive
example $x$, when compared with examples from other group $j\neq
g(i)$, satisfies
\[
\langle u_{g(i)},x\rangle - \langle u_j,x\rangle
=
\langle u_{g(i)}-u_j,x\rangle 
\geq \gamma,
\]
implying that all examples in class $i$ lie in
\[
R^{+}_i = \bigcap_{j\neq g(i)}\{x : \langle u_{g(i)}-u_j,x\rangle \geq \gamma\}, 
\]
while all examples in other groups lie in
\[
R^{-}_i = \bigcup_{j\neq g(i)}\{x : \langle u_{g(i)}-u_j,x\rangle \leq -\gamma\}.
\]
When comparing with other classes $j$ in the same group $g(i)$, from
Lemma~\ref{lemma:boundaries}, we know that there exists thresholds
$b_i$ and $t_i$ that can be used to separate examples from group $i$,
i.e., all its positive examples lie in 
\[
\hat{R}^{+}_i=\{x : \langle u'_{g(i)},x\rangle \geq b_i+\gamma\}
\cap
\{x : \langle u'_{g(i)},x\rangle \leq t_i-\gamma\},
\]
while examples from other classes in group $g(i)$ lie in 
\[
\hat{R}^{-}_i=\{x : \langle u'_{g(i)},x\rangle \leq b_i-\gamma\}
\cup
\{x : \langle u'_{g(i)},x\rangle \geq t_i+\gamma\}.
\]
Let $v_b=\frac{b_i}{\|u'_{g(i)}\|}u'_{g(i)}$ and
$v_t=\frac{t_i}{\|u'_{g(i)}\|}u'_{g(i)}$.  Both sets can be expressed as
\begin{align*}
\hat{R}^{+}_i = & \ 
\{x : \langle u'_{g(i)},x\rangle \geq \langle u'_{g(i)},v_b \rangle+\gamma\} 
\ \cap \\
& \ \{x : \langle u'_{g(i)},x\rangle \leq \langle u'_{g(i)},v_t \rangle-\gamma\},
\end{align*}
while examples from other classes in group $g(i)$ lie in 
\begin{align*}
\hat{R}^{-}_i = & \
\{x : \langle u'_{g(i)},x\rangle \leq \langle u'_{g(i)},v_b \rangle-\gamma\}
\ \cup \\
& \ \{x : \langle u'_{g(i)},x\rangle \geq \langle u'_{g(i)},v_t \rangle+\gamma\}.
\end{align*}

From Lemma~\ref{lem:norm-bound}, for class $i$, it is enough to
establish a multivariate polynomial $p_i$ such that
\begin{align*}
x\in R^{+}_i \cap \hat{R}^{+}_i & \ \ \ \Rightarrow & p_i(x) & \geq \gamma'/2, \\
x\in R^{-}_i \cup \hat{R}^{-}_i & \ \ \ \Rightarrow & p_i(x) & \leq -\gamma'/2.
\end{align*}


This is shown in the Theorem~\ref{thm:sep-poly} below.  This theorem
is fairly technical and is proved in Literature~\ref{sect:sep-poly}

\begin{theorem}
(Polynomial approximation of intersection of halfspaces)
Let $v_1,v_2,\ldots,v_m \in V$ such that $\|v_1\|,\|v_2\|,\ldots,\|v_m\| \leq 1$.
Let $v_b,v_t\in V$ such that $\|v_b\|\leq 1$ and $\|v_t\|\leq 1$.
Let $v' \in V$ such that $\|v'\|\leq 1$.
Let $\gamma \in (0,1)$ and $x \in \ball(0,1)$.
There exists a multivariate polynomial $p:\R^d\to \R$ such that
\begin{enumerate}
\item $p(x) \geq \frac{1}{2}$ for all $x \in \left(\bigcap_{i=1}^m \left\{x: \fdot{v_i}{x}\geq \gamma\right\}\right) \cap \left\{x:\fdot{x}{v'} \geq \fdot{v_b}{v'}+\gamma\right\} \cap \left\{x: \fdot{x}{v'} \leq \fdot{v_t}{v'}-\gamma\right\},$
\item $p(x) \leq -\frac{1}{2}$ for all $x \in \left(\bigcup_{i=1}^m \left\{x: \fdot{v_i}{x}\leq -\gamma\right\}\right) \cup \left\{x: \fdot{x}{v'} \leq \fdot{v_b}{v'}-\gamma\right\} \cup \left\{x: \fdot{x}{v'} \geq \fdot{v_t}{v'}+\gamma\right\},$
\item $deg(p)=\lceil\log_2(2m+4)\rceil\cdot\left\lceil\sqrt{\frac{2}{\gamma}}\right\rceil,$
\item $\|p\|\leq \frac{9}{2}\left[420\lceil\log_2(2m+4)\rceil\cdot\left\lceil\sqrt{\frac{2}{\gamma}}\right\rceil\right]^{\frac{\lceil\log_2(2m+4)\rceil\cdot\left\lceil\sqrt{\frac{2}{\gamma}}\right\rceil}{2}}$
\end{enumerate}

\label{thm:sep-poly}
\end{theorem}

Our proof follows the approach in~\cite{BeygelzimerPSTWZ2019-separable}.

\begin{proof}[Proof of Theorem~\ref{thm:sep-poly}]
Let $r=\lceil\log_2(2m+4)\rceil$ and $s=\left\lceil\sqrt{\frac{2}{\gamma}}\right\rceil$.
Define the polynomial $p:\R^d\to \R$ as
\begin{align*}
    p(x) &= m+\frac{5}{2}-\sum_{i=1}^m (T_s(1-\fdot{v_i}{x}))^r \\
    &-(T_s(1-\fdot{x-v_b}{v'}/2))^r \\
    &-(T_s(1-\fdot{v_t-x}{v'}/2))^r.
\end{align*}

First, consider the case when
\begin{align*}
  x \in & \left(\bigcap_{i=1}^m \left\{x: \fdot{v_i}{x}\geq \gamma\right\}\right) \cap \left\{x: \fdot{x}{v'} \geq \fdot{v_b}{v'}+\gamma\right\} \cap \\
  & \left\{x: \fdot{x}{v'} \leq \fdot{v_t}{v'}-\gamma\right\}.
\end{align*}

Note that $\langle v_i, x\rangle \geq \gamma$ for all $i\in [m]$.
Since $\|x\|\leq 1$ and $\|v_i\|\leq 1$, we have $\fdot{v_i}{x} \in [0,1]$; 
thus, $(T_s(1-\fdot{v_i}{x}))^r \in [-1,1]$.
Consider the terms involving $v_b$ and $v_t$.  Since $\|x\|,\|v_b\|,\|v_t\|\leq 1$, we have that $\|x-v_b\|\leq 2$ and $\|v_t-x\|\leq 2$.  This implies that
$1\geq\fdot{x-v_b}{v'}/2\geq\gamma/2$ and $1\geq\fdot{v_t-x}{v'}/2\geq\gamma/2$; hence,
$(T_s(1-\fdot{x-v_b}{v'}/2))^r\in [-1,1]$ and $(T_s(1-\fdot{v_t-x}{v'}/2))^r\in[-1,1]$.
Therefore,
\[
p(x)\geq m+\frac{5}{2}-m-1-1\geq \frac{1}{2}.
\]

Now consider the case when
\begin{align*}
  x  \in & \bigcup_{i=1}^m \left\{x:\fdot{v_i}{x}\leq -\gamma\right\} \cup \left\{x:\fdot{x}{v'} \leq \fdot{v_b}{v'}-\gamma\right\} \cup \\
  & \left\{x:\fdot{x}{v'} \geq \fdot{v_t}{v'}+\gamma\right\}
\end{align*}
There are two subcases to consider.

{\em Subcase 1:} Suppose that for some $i$, $\fdot{v_i}{x}\leq-\gamma$.  
In this case, $1-\fdot{v_i}{x}\geq 1+\gamma$ and
Lemma~\ref{lem:cheby-prop} (part 6) implies that
\[
  T_s(1-\fdot{v_i}{x}) \geq 1+s^2\gamma \geq 1+2 \geq 2,
\]
and thus, $(T_s(1-\fdot{v_i}{x}))^r\geq 2^r\geq 2m+4$.

Since $T_s(1-\fdot{v_i}{x}))^r \geq -1$ for all $i$, 
$(T_s(1-\fdot{x-v_b}{v'}/2))^r\geq -1$, and $(T_s(1-\fdot{v_t-x}{v'}/2))^r\geq -1$, 
we have that
\begin{align*}
    p(x)&=m+\frac{5}{2}-(T_s(1-\fdot{v_i}{x}))^r \\
    &-\sum_{j\in [m]j\neq i} (T_s(1-\fdot{v_j}{x}))^r \\
    &-(T_s(1-\fdot{x-v_b}{v'}/2))^r \\
    &-(T_s(1-\fdot{v_t-x}{v'}/2))^r \\
    &\leq m+\frac{5}{2}-(2m+4)+(m-1)+2 \leq -\frac{1}{2}.
\end{align*}

{\em Subcase 2:} Consider the other case when for all $i$, $\fdot{v_i}{x} > -\gamma$.
We deal with the case that $\fdot{x}{v'} \leq \fdot{v_b}{v'}-\gamma$.  
The case when $\fdot{x}{v'} \geq \fdot{v_t}{v'}+\gamma$ can be handled similarly.

Since $\fdot{x-v_b}{v'} \leq -\gamma$, we have $1-\fdot{x-v_b}{v'}/2\geq 1 + \gamma/2$.
Lemma~\ref{lem:cheby-prop} (part 6) implies that
\[
  T_s(1-\fdot{x-v_b}{v'}/2)\geq 1+s^2\gamma/2 \geq 1+2/2 \geq 2,
\]
and $(T_s(1-\fdot{x-v_b}{v'}/2))^r\geq 2m+4$.  
Applying the same argument as in Subcase 1, this implies that $p(x)\leq-\frac{1}{2}$.

The degree of $p$ is the maximum degree of the terms $(T_s(1-\fdot{v_i}{x}))^r$, $(T_s(1-\fdot{x-v_b}{v'}/2))^r$, and $(T_s(1-\fdot{v_t-x}{v'})/2)^r$; thus, it is $r\cdot s$.

Finally, we prove the upper bound of norm of $p$. 
Let $f_i(x)=1-\fdot{v_i}{x}$, 
let $k_b(x)=1-\fdot{x-v_b}{v'}/2$, $k_t(x)=1-\fdot{v_t-x}{v'}/2$.
Notice that $\fdot{v_i}{x}$,$\fdot{x-v_b}{v'}$ and $\fdot{v_t-x}{v'}$ are multivariate polynomial.
Let $h_i(x)=\fdot{v_i}{x}$ then $f_i(x)=1-h_i(x)$, and use property 3 of lemma~\ref{lem:poly-prop} we have
\begin{align*}
% \|f_i\|^2\leq 1+\|v_i\|^2\cdot\|x\|^2\leq 1+1=2, \\
\|f_i\|^2&\leq 2\left(\|1\|^2 +\|h_i(x)\|^2 \right) \\
&\leq 2\left(\|1\|^2 +\|v_i\|^2 \right), & \text{part 2 of lemma~\ref{lem:poly-prop}} \\
&\leq 2(1+1) \\
&=4 
\end{align*}
Let $h'(x)=\fdot{x}{v'}$ then $k_b(x)=1-\fdot{x-v_b}{v'}/2=1-h'(x)/2+h'(v_b)/2$
\begin{align*}
\|k_b\|^2&\leq 3\left( \|1\|^2 + \frac{\|h'(x)\|^2}{2}+ \frac{\|h'(v_b)\|^2}{2} \right) \\
&\leq 3\left( \|1\|^2 + \frac{\|v'\|^2}{2}+ \frac{\|v'\|^2}{2} \right) & \text{part 2 of lemma~\ref{lem:poly-prop}} \\
&= 3\left(1+\frac{1}{2}+\frac{1}{2}\right) \\
&=6,
\end{align*}
and also $k_t(x)=1-\fdot{v_t-x}{v'}/2=1-h'(v_t)/2+h'(x)/2$ then
\begin{align*}
  \|k_t\|^2&\leq 3\left( \|1\|^2 + \frac{\|h'(v_t)\|^2}{2}+ \frac{\|h'(x)\|^2}{2} \right)=6.
\end{align*}

Let $T_s(z)=\sum_{j=0}^s c_j z^j$ be the expansion of $s$-th Chebyshev polynomial.

We first deal with $\|T_s(1-\fdot{v_i}{x})\|^2$.
By lemma~\ref{lem:cheby-prop} and \ref{lem:poly-prop}, $s+1\leq2^s$ for any non-negative integer, we have
% need prop of norm of polynomial
\begin{align*}
    \|T_s(1-\fdot{v_i}{x})\|^2 &= \|T_s(f_i)\|^2 \\
    &=\left\|\sum_{j=0}^s c_j(f_i)^j\right\|^2\\
    &\leq (s+1)\sum_{j=0}^s\left\| c_j(f_i)^j\right\|^2 & &\text{part 3 of lemma~\ref{lem:poly-prop}}\\
    &=(s+1)\sum_{j=0}^s c_j^2\left\|(f_i)^j\right\|^2 \\
    &\leq (s+1)\sum_{j=0}^s c_j^2j^j\left\|f_i\right\|^{2j} & &\text{part 2 of lemma~\ref{lem:poly-prop}}\\
    &\leq (s+1)\sum_{j=0}^s c_j^2j^j4^{2j}\\
    &\leq (s+1)s^s4^{2s}\sum_{j=0}^s c_j^2 \\
    &=(s+1)s^s4^{2s}\|T_s\|^2 \\
    &=(s+1)s^s4^{2s}(1+\sqrt{2})^{2s} & &\text{part 7 of lemma~\ref{lem:cheby-prop}}\\
    &=(s+1)\left(16(1+\sqrt{2})^2s\right)^s \\
    &\leq (32(1+\sqrt{2})^2s)^s \\
    &\leq (187s)^s.
\end{align*}

The other two terms $\|T_s(\fdot{x-v_b}{v'}/2)\|^2$ and $\|T_s(\fdot{v_t-x}{v'}/2)\|^2$ can be analyzed similarly.  We have that
\begin{align*}
    \|T_s(\fdot{x-v_b}{v'}/2)\|^2&=\|T_s(k_b)\|^2 \\
    &=\left\|\sum_{j=0}^s c_j(k_b)^j\right\|^2 \\
    &\leq (s+1)\sum_{j=0}^s c_j^2j^j\left\|k_b\right\|^{2j} & &\text{part 2,3 of lemma~\ref{lem:poly-prop}}\\
    &\leq (s+1)\sum_{j=0}^s c_j^2j^j6^{2j} \\
    &\leq (s+1)s^s6^{2s}\sum_{j=0}^s c_j^2 \\
    &=(s+1)s^s6^{2s}\|T_s\|^2 \\
    &=(s+1)s^s6^{2s}(1+\sqrt{2})^{2s} & &\text{part 7 of lemma~\ref{lem:cheby-prop}} \\
    &=(s+1)\left(36(1+\sqrt{2})^2s\right)^s \\
    &\leq (72(1+\sqrt{2})^2s)^s \\
    &\leq (420s)^s
\end{align*}
% (105s')^s'
and
\begin{align*}
    \|T_s(\fdot{v_t-x}{v'}/2)\|^2 &= \|T_s(k_t)\|^2 \\
    &=\left\|\sum_{j=0}^s c_j(k_t)^j\right\|^2 \\
    &\leq (420s)^s.
\end{align*}

Finally,
\begin{align*}
    \|p\|&\leq m+\frac{5}{2}+\sum_{i=1}^m\left\|T_s(f_i)^r\right\|
    +\left\|T_s(k_b)^r\right\|+\left\|T_s(k_t)^r\right\| \\
    &=m+\frac{5}{2}+\sum_{i=1}^m\sqrt{\left\|T_s(f_i)^r\right\|^2} \\
    &\ \ \ +\sqrt{\left\|T_s(k_b)^r\right\|^2} +\sqrt{\left\|T_s(k_t)^r\right\|^2} \\
    &\leq m+\frac{5}{2}+\sum_{i=1}^m\sqrt{r^{rs}\left\|T_s(f_i)^r\right\|^{2r}} \\
    &\ \ \ +\sqrt{r^{rs}\left\|T_s(k_b)^r\right\|^{2r}}+\sqrt{r^{rs}\left\|T_s(k_t)^r\right\|^{2r}} \\
    &\leq m+\frac{5}{2}+m r^{rs/2}(187s)^{rs/2}+r^{rs/2}(420s)^{rs/2} \\ 
    &\ \ \ +r^{rs/2}(420s)^{rs/2} \\
    &\leq m+\frac{5}{2}+(m+2)(420rs)^{rs/2}.
\end{align*}
Using the fact that $m\leq\frac{1}{2}2^r$ and $r,s \geq 1$, we then have
\begin{align*}
    \|p\| &\leq m+\frac{5}{2}+(m+2)(420rs)^{rs/2} \\
        &\leq \frac{1}{2}2^r+\frac{5}{2}+\left( \frac{1}{2}2^r +2\right)(420rs)^{rs/2} \\
        &\leq 2\cdot2^r+\frac{5}{2}\cdot 2^r(420rs)^{rs/2} \\
        &= 2^r\left( 2+\frac{5}{2}\right)(420rs)^{rs/2} \\
        &\leq 4^{rs/2}\cdot\frac{9}{2}(420rs)^{rs/2} \\
        &=\frac{9}{2}(1680rs)^{rs/2}.
\end{align*}
Substitutions of $r$ and $s$ finish the proof.
\end{proof}

Our main technical result is the following margin transformation using the rational kernel.

\begin{theorem}
(Margin transformation). Let $(x_1,y_1),(x_2,y_2),\ldots,(x_T,y_T)\in \ball(0,1)\times [K]$
be a sequence of labeled examples that is group weakly linear separable with margin $\gamma >0$.
Let $L$ be number of group weakly separable such that $L\leq K.$
Let $\phi$ defined as in (\ref{eqn:phi}) let
\[
    \gamma' = \frac{\left[3360\lceil\log_2(2L+2)\rceil\cdot\left\lceil\sqrt{\frac{2}{\gamma}}\right\rceil\right]^{-\frac{\lceil\log_2(2L+2)\rceil\cdot\left\lceil\sqrt{\frac{2}{\gamma}}\right\rceil}{2}}}{9\sqrt{L}},
\]
The feature map $\phi$ makes the sequence $(\phi (x_1),y_1),(\phi (x_2),y_2),\ldots,(\phi (x_T),y_T)$
strongly linearly separable with margin $\gamma'$.
\label{thm:margin-trans}
\end{theorem}

We note that the margin depends on $L$, the number of groups, instead of $K$, the number of classes.  Using Theorem~\ref{thm:margin-trans} with Theorem~\ref{thm:kernel-bandit-mistake-bound} we obtain the following mistake bound for our algorithm.

\begin{proof}[Proof of Theorem~\ref{thm:margin-trans}]

Consider class $i\in[K]$. We will apply
Theorem~\ref{thm:sep-poly}. For $j\in\{1,\ldots,L-1\}$, let
\[
v_j=\left\{
\begin{array}{ll}
    u_{g(i)}-u_j, & \mbox{if $j<g(i)$,} \\
    u_{g(i)}-u_{j+1}, & \mbox{if $j>g(i)$.}
\end{array}
\right.    
\]
Also, let $v'=u'_{g(i)}$, $v_b=\frac{b_i}{\|u'_{g(i)}\|}u'_{g(i)}$
and $v_t=\frac{t_i}{\|u'_{g(i)}\|}u'_{g(i)}$.

From Theorem~\ref{thm:sep-poly}, there exists a multivariate
polynomial $p_i:\R^d\to \R$ such that for all $t\in[T]$ and the
sequence $(x_1,y_1),(x_2,y_2),(x_t,y_t),\ldots,(x_T,y_T)$, we have
\begin{itemize}
\item if $y_t=i$, $p_i(x_t)\geq \frac{1}{2}$, since $x_t\in R^{+}_i
    \cap \hat{R}^{+}_i$, and
\item if $y_t\neq i,$ $p_i(x_t)\leq -\frac{1}{2}$, since $x_t\in
    R^{-}_i \cap \hat{R}^{-}_i$.
\end{itemize}

It is left to check the properties of $p$.
Theorem~\ref{thm:sep-poly} implies that
\[
\|p\|\leq \frac{9}{2}\left[1680\lceil\log_2(2L+2)\rceil\cdot\left\lceil\sqrt{\frac{2}{\gamma}}\right\rceil\right]^{\frac{\lceil\log_2(2L+2)\rceil\cdot\left\lceil\sqrt{\frac{2}{\gamma}}\right\rceil}{2}}
\]
By Lemma~\ref{lem:norm-bound}, there exists $c_i\in\ell_2$ such that $\fdot{c_i}{\phi(x)}=p_i(x),$ and
\[
\|c_i\|_{\ell_2}\leq \frac{9}{2}\left[3360\lceil\log_2(2L+2)\rceil\cdot\left\lceil\sqrt{\frac{2}{\gamma}}\right\rceil\right]^{\frac{\lceil\log_2(2L+2)\rceil\cdot\left\lceil\sqrt{\frac{2}{\gamma}}\right\rceil}{2}}.
\]

We are ready to construct strongly separable vectors for our group
weakly separable case in $\ell_2$ such that
$\|z_1\|^2+\|z_2\|^2+\ldots+\|z_L\|^2\leq 1$ and for all $t\in [T]$,
$\fdot{z_{y_t}}{x_t} \geq \gamma$, and for all $j\neq y_t$,
$\fdot{z_j}{x_t}\leq -\gamma$, by scaling $c_i$ appropriately as
follows.  We can let
\[
z_i=\frac{c_i}{\sqrt{L}\cdot \frac{9}{2}\left[3360\lceil\log_2(2L+2)\rceil\cdot\left\lceil\sqrt{\frac{2}{\gamma}}\right\rceil\right]^{\frac{\lceil\log_2(2L+2)\rceil\cdot\left\lceil\sqrt{\frac{2}{\gamma}}\right\rceil}{2}}},
\]
and
\[
\gamma = \frac{\left[3360\lceil\log_2(2L+2)\rceil\cdot\left\lceil\sqrt{\frac{2}{\gamma}}\right\rceil\right]^{-\frac{\lceil\log_2(2L+2)\rceil\cdot\left\lceil\sqrt{\frac{2}{\gamma}}\right\rceil}{2}}}{9\sqrt{L}},
\]
then the theorem follows.    
\end{proof}

\begin{corollary}
(Mistake bound for group weakly linearly separable case) 
Let $K$ be positive integer, $L\leq K$ and $\gamma$ be positive real number. 
The mistake bound made by Algorithm~\ref{alg:kernel-bandit} when the examples are group weakly
linearly separable with margin $\gamma$ with $L$ groups is at most
$K\cdot 2^{\tilde{O}(\sqrt{1/\gamma}\log L)}$.
\label{cor:mistake-bound}
\end{corollary}

Note that multiplicative factor of $K$ is hidden from the second bound of~\cite{BeygelzimerPSTWZ2019-separable} because of the $\tilde{O}$ notation on the exponent.  We cannot do that because in our exponent we have only $\log L$ which can be much smaller than $K$.  Their actual bound (showing $K$), which can be compared to ours, is $K\cdot 2^{\tilde{O}(\sqrt{1/\gamma}\log K)}$.
  